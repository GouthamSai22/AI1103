\documentclass[journal,12pt,twocolumn]{IEEEtran}

\usepackage{setspace}
\usepackage{gensymb}
\singlespacing
\usepackage[cmex10]{amsmath}

\usepackage{amsthm}

\usepackage{mathrsfs}
\usepackage{txfonts}
\usepackage{stfloats}
\usepackage{bm}
\usepackage{cite}
\usepackage{cases}
\usepackage{subfig}

\usepackage{longtable}
\usepackage{multirow}

\usepackage{enumitem}
\usepackage{mathtools}
\usepackage{steinmetz}
\usepackage{tikz}
\usepackage{circuitikz}
\usepackage{verbatim}
\usepackage{tfrupee}
\usepackage[breaklinks=true]{hyperref}
\usepackage{graphicx}
\usepackage{tkz-euclide}

\usetikzlibrary{calc,math}
\usepackage{listings}
    \usepackage{color}                                            %%
    \usepackage{array}                                            %%
    \usepackage{longtable}                                        %%
    \usepackage{calc}                                             %%
    \usepackage{multirow}                                         %%
    \usepackage{hhline}                                           %%
    \usepackage{ifthen}                                           %%
    \usepackage{lscape}     
\usepackage{multicol}
\usepackage{chngcntr}

\DeclareMathOperator*{\Res}{Res}

\renewcommand\thesection{\arabic{section}}
\renewcommand\thesubsection{\thesection.\arabic{subsection}}
\renewcommand\thesubsubsection{\thesubsection.\arabic{subsubsection}}

\renewcommand\thesectiondis{\arabic{section}}
\renewcommand\thesubsectiondis{\thesectiondis.\arabic{subsection}}
\renewcommand\thesubsubsectiondis{\thesubsectiondis.\arabic{subsubsection}}


\hyphenation{op-tical net-works semi-conduc-tor}
\def\inputGnumericTable{}                                 %%

\lstset{
%language=C,
frame=single, 
breaklines=true,
columns=fullflexible
}
\begin{document}

\newcommand{\BEQA}{\begin{eqnarray}}
\newcommand{\EEQA}{\end{eqnarray}}
\newcommand{\define}{\stackrel{\triangle}{=}}
\bibliographystyle{IEEEtran}
\raggedbottom
\setlength{\parindent}{0pt}
\providecommand{\mbf}{\mathbf}
\providecommand{\pr}[1]{\ensuremath{\Pr\left(#1\right)}}
\providecommand{\qfunc}[1]{\ensuremath{Q\left(#1\right)}}
\providecommand{\sbrak}[1]{\ensuremath{{}\left[#1\right]}}
\providecommand{\lsbrak}[1]{\ensuremath{{}\left[#1\right.}}
\providecommand{\rsbrak}[1]{\ensuremath{{}\left.#1\right]}}
\providecommand{\brak}[1]{\ensuremath{\left(#1\right)}}
\providecommand{\lbrak}[1]{\ensuremath{\left(#1\right.}}
\providecommand{\rbrak}[1]{\ensuremath{\left.#1\right)}}
\providecommand{\cbrak}[1]{\ensuremath{\left\{#1\right\}}}
\providecommand{\lcbrak}[1]{\ensuremath{\left\{#1\right.}}
\providecommand{\rcbrak}[1]{\ensuremath{\left.#1\right\}}}
\theoremstyle{remark}
\newtheorem{rem}{Remark}
\newcommand{\sgn}{\mathop{\mathrm{sgn}}}
\providecommand{\abs}[1]{\vert#1\vert}
\providecommand{\res}[1]{\Res\displaylimits_{#1}} 
\providecommand{\norm}[1]{\lVert#1\rVert}
%\providecommand{\norm}[1]{\lVert#1\rVert}
\providecommand{\mtx}[1]{\mathbf{#1}}
\providecommand{\mean}[1]{E[ #1 ]}
\providecommand{\fourier}{\overset{\mathcal{F}}{ \rightleftharpoons}}
%\providecommand{\hilbert}{\overset{\mathcal{H}}{ \rightleftharpoons}}
\providecommand{\system}{\overset{\mathcal{H}}{ \longleftrightarrow}}
	%\newcommand{\solution}[2]{\textbf{Solution:}{#1}}
\newcommand{\solution}{\noindent \textbf{Solution: }}
\newcommand{\cosec}{\,\text{cosec}\,}
\providecommand{\dec}[2]{\ensuremath{\overset{#1}{\underset{#2}{\gtrless}}}}
\newcommand{\myvec}[1]{\ensuremath{\begin{pmatrix}#1\end{pmatrix}}}
\newcommand{\mydet}[1]{\ensuremath{\begin{vmatrix}#1\end{vmatrix}}}
\numberwithin{equation}{subsection}
\makeatletter
\@addtoreset{figure}{problem}
\makeatother
\let\StandardTheFigure\thefigure
\let\vec\mathbf
\renewcommand{\thefigure}{\theproblem}
\def\putbox#1#2#3{\makebox[0in][l]{\makebox[#1][l]{}\raisebox{\baselineskip}[0in][0in]{\raisebox{#2}[0in][0in]{#3}}}}
     \def\rightbox#1{\makebox[0in][r]{#1}}
     \def\centbox#1{\makebox[0in]{#1}}
     \def\topbox#1{\raisebox{-\baselineskip}[0in][0in]{#1}}
     \def\midbox#1{\raisebox{-0.5\baselineskip}[0in][0in]{#1}}
\vspace{3cm}
\title{Assignment 1}
\author{S Goutham Sai - CS20BTECH11042}
\maketitle
\newpage
\bigskip
\renewcommand{\thefigure}{\theenumi}
\renewcommand{\thetable}{\theenumi}
Download all python codes from 
\begin{lstlisting}
https://github.com/GouthamSai22/AI1103/blob/main/Assignment1/Codes/assign1.py
\end{lstlisting}
%
and latex-tikz codes from 
%
\begin{lstlisting}
https://github.com/GouthamSai22/AI1103/blob/main/Assignment1/main.tex
\end{lstlisting}
\begin{center}
  \section{\textbf{Problem Statement}} 
\end{center}
In a game, a man wins a rupee for a six and loses a rupee for any other number when a fair die is thrown. The man decided to throw a dice twice but to quit as and when he gets a six. Find the expected value of the amount he wins or loses.
\begin{center}
   \section{\textbf{Solution}}
\end{center}
Let $X \epsilon \{0,1,2,3\}$ represent a random variable where
\begin{itemize}
    \item 0 $\rightarrow$ man wins in $1^{st}$ roll.
    \item 1 $\rightarrow$ man wins in $2^{st}$ roll.
    \item 2 $\rightarrow$ man wins in $3^{st}$ roll.
    \item 3 $\rightarrow$ man lost in all 3 rolls.
\end{itemize}
Given, probability of man winning any round is $\frac{1}{6}$ and hence probability of him losing any round is $\frac{5}{6}$
\begin{center}
    \begin{tabular}{|c|c|c|}
      \hline
      i & & P(X=i)  \\
      \hline
      0 & $\frac{1}{6}$ & $\frac{1}{6}$ \\[1 ex]
      \hline
      1 & $\frac{5}{6} \times \frac{1}{6}$ & $\frac{5}{36}$ \\[1 ex]
      \hline
      2 & $\frac{5}{6} \times \frac{5}{6} \times \frac{1}{6}$ & $\frac{25}{216}$ \\[1 ex]
      \hline
      3 & $\frac{5}{6} \times \frac{5}{6} \times \frac{5}{6}$ & $\frac{125}{216}$ \\[1 ex]
      \hline
    \end{tabular}
\end{center}
Since the man gets a rupee for every win and gives a rupee for every loss, the expected value of amount gained is
\begin{multline}
    Expected \ value = \sum_{i=0}^{3} \pr{X=i} \times \\(Amount \ gained \ when \ X=i)
\end{multline}
\begin{multline}
    Expected \ value &= \brak{\frac{1}{6} \times 1}   + \left( \frac{5}{36} \times 0 \right) + \left(\frac{25}{216} \times (-1) \right)\\
    + \left(\frac{125}{216} \times (-3) \right) 
\end{multline}
\begin{align}
    Expected \ value &= \frac{1}{6} - \frac{25}{216} - \frac{375}{216} \\
    &= \frac{-364}{216} 
\end{align}
\begin{equation}
    \boxed{Expected \ Value = -1.685}
\end{equation}
  
\end{document}