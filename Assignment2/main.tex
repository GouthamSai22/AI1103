\documentclass[journal,12pt,twocolumn]{IEEEtran}

\usepackage{setspace}
\usepackage{gensymb}
\singlespacing
\usepackage[cmex10]{amsmath}

\usepackage{amsthm}

\usepackage{mathrsfs}
\usepackage{txfonts}
\usepackage{stfloats}
\usepackage{bm}
\usepackage{cite}
\usepackage{cases}
\usepackage{subfig}
\usepackage{paralist}
\usepackage{longtable}
\usepackage{multirow}

\usepackage{enumitem}
\usepackage{mathtools}
\usepackage{steinmetz}
\usepackage{tikz}
\usepackage{circuitikz}
\usepackage{verbatim}
\usepackage{tfrupee}
\usepackage[breaklinks=true]{hyperref}
\usepackage{graphicx}
\usepackage{tkz-euclide}

\usetikzlibrary{calc,math}
\usepackage{listings}
    \usepackage{color}                                            %%
    \usepackage{array}                                            %%
    \usepackage{longtable}                                        %%
    \usepackage{calc}                                             %%
    \usepackage{multirow}                                         %%
    \usepackage{hhline}                                           %%
    \usepackage{ifthen}                                           %%
    \usepackage{lscape}     
\usepackage{multicol}
\usepackage{chngcntr}

\DeclareMathOperator*{\Res}{Res}

\renewcommand\thesection{\arabic{section}}
\renewcommand\thesubsection{\thesection.\arabic{subsection}}
\renewcommand\thesubsubsection{\thesubsection.\arabic{subsubsection}}

\renewcommand\thesectiondis{\arabic{section}}
\renewcommand\thesubsectiondis{\thesectiondis.\arabic{subsection}}
\renewcommand\thesubsubsectiondis{\thesubsectiondis.\arabic{subsubsection}}


\hyphenation{op-tical net-works semi-conduc-tor}
\def\inputGnumericTable{}                                 %%

\lstset{
%language=C,
frame=single, 
breaklines=true,
columns=fullflexible
}
\begin{document}

\newcommand{\BEQA}{\begin{eqnarray}}
\newcommand{\EEQA}{\end{eqnarray}}
\newcommand{\define}{\stackrel{\triangle}{=}}
\bibliographystyle{IEEEtran}
\raggedbottom
\setlength{\parindent}{0pt}
\providecommand{\mbf}{\mathbf}
\providecommand{\pr}[1]{\ensuremath{\Pr\left(#1\right)}}
\providecommand{\qfunc}[1]{\ensuremath{Q\left(#1\right)}}
\providecommand{\sbrak}[1]{\ensuremath{{}\left[#1\right]}}
\providecommand{\lsbrak}[1]{\ensuremath{{}\left[#1\right.}}
\providecommand{\rsbrak}[1]{\ensuremath{{}\left.#1\right]}}
\providecommand{\brak}[1]{\ensuremath{\left(#1\right)}}
\providecommand{\lbrak}[1]{\ensuremath{\left(#1\right.}}
\providecommand{\rbrak}[1]{\ensuremath{\left.#1\right)}}
\providecommand{\cbrak}[1]{\ensuremath{\left\{#1\right\}}}
\providecommand{\lcbrak}[1]{\ensuremath{\left\{#1\right.}}
\providecommand{\rcbrak}[1]{\ensuremath{\left.#1\right\}}}
\theoremstyle{remark}
\newtheorem{rem}{Remark}
\newcommand{\sgn}{\mathop{\mathrm{sgn}}}
\providecommand{\abs}[1]{\vert#1\vert}
\providecommand{\res}[1]{\Res\displaylimits_{#1}} 
\providecommand{\norm}[1]{\lVert#1\rVert}
%\providecommand{\norm}[1]{\lVert#1\rVert}
\providecommand{\mtx}[1]{\mathbf{#1}}
\providecommand{\mean}[1]{E[ #1 ]}
\providecommand{\fourier}{\overset{\mathcal{F}}{ \rightleftharpoons}}
%\providecommand{\hilbert}{\overset{\mathcal{H}}{ \rightleftharpoons}}
\providecommand{\system}{\overset{\mathcal{H}}{ \longleftrightarrow}}
	%\newcommand{\solution}[2]{\textbf{Solution:}{#1}}
\newcommand{\solution}{\noindent \textbf{Solution: }}
\newcommand{\cosec}{\,\text{cosec}\,}
\providecommand{\dec}[2]{\ensuremath{\overset{#1}{\underset{#2}{\gtrless}}}}
\newcommand{\myvec}[1]{\ensuremath{\begin{pmatrix}#1\end{pmatrix}}}
\newcommand{\mydet}[1]{\ensuremath{\begin{vmatrix}#1\end{vmatrix}}}
\numberwithin{equation}{subsection}
\makeatletter
\@addtoreset{figure}{problem}
\makeatother
\let\StandardTheFigure\thefigure
\let\vec\mathbf
\renewcommand{\thefigure}{\theproblem}
\def\putbox#1#2#3{\makebox[0in][l]{\makebox[#1][l]{}\raisebox{\baselineskip}[0in][0in]{\raisebox{#2}[0in][0in]{#3}}}}
     \def\rightbox#1{\makebox[0in][r]{#1}}
     \def\centbox#1{\makebox[0in]{#1}}
     \def\topbox#1{\raisebox{-\baselineskip}[0in][0in]{#1}}
     \def\midbox#1{\raisebox{-0.5\baselineskip}[0in][0in]{#1}}
\vspace{3cm}
\title{Assignment 1}
\author{S Goutham Sai - CS20BTECH11042}
\maketitle
\newpage
\bigskip
\renewcommand{\thefigure}{\theenumi}
\renewcommand{\thetable}{\theenumi}
Download all python codes from 
\begin{lstlisting}
https://github.com/GouthamSai22/AI1103/blob/main/Assignment2/Codes
\end{lstlisting}
%
and latex-tikz codes from 
%
\begin{lstlisting}
https://github.com/GouthamSai22/AI1103/blob/main/Assignment2/main.tex
\end{lstlisting}
\section{\textbf{Problem 70 from GATE EC}}
Let X and Y be continuous random variables with the joint probabilitiy distribution function\newline
f(x,y) = $\bigg\{ \begin{array}{lr}
    ae^{-2y}, & 0<x<y<\infty  \\
     0, &\text{otherwise} 
\end{array}$\newline
The value of $E(X|Y=2)$ is \newline
\begin{inparaenum}[(A)]
\item 4 \hspace{1cm}
\item 3 \hspace{1cm}
\item 2 \hspace{1cm}
\item 1 \hspace{1cm}
\end{inparaenum}
\section{\textbf{Solution}}
Given two continuous random variables X and Y, whose joint probability distribution function is\newline
\begin{equation}
    f(x,y) = \bigg\{ \begin{array}{lr}
    ae^{-2y}, & 0<x<y<\infty  \\
     0, &\text{otherwise} 
\end{array}
\end{equation}
We are asked to find the value of $E(X|Y=2)$. Since Y=2, we can say that the probability distribution function of x in this case is
\begin{equation}
    \pr{X=x} = \bigg\{ \begin{array}{lr}
         ae^{-2\times 2}, &0<x<2  \\
         0 & \text{otherwise} 
    \end{array}
\end{equation}
From the properties of the joint probability function,
\begin{equation}
    \iint_{-\infty}^{\infty}f(x,y) \,dx\,dy = 1
\end{equation}
From the definition of the joint probability equation, \equationautorefname{2.0.3} can be written as
\begin{align}
    \int_0^{\infty}\brak{\int_0^y ae^{-2y} \,dx} \,dy &= 1 \\
    \int_0^{\infty}\brak{ae^{-2y}\int_0^y 1 \,dx} \,dy &=1 \\
    \int_0^{\infty}\brak{ae^{-2y}y} \,dy &= 1 \\
    a\int_0^{\infty} e^{-2y}y \,dy &= 1 \\
    a\brak{-\frac{1}{2}e^{-2y}\brak{y+\frac{1}{2}}}_0^{\infty} &= 1 \\
    a \brak{\frac{1}{4}} &= 1 \\
    \Rightarrow a &= 4
\end{align}
Hence, \equationautorefname{2.0.2} can be written as
\begin{equation}
    \pr{X=x} = \bigg\{ \begin{array}{lr}
         4e^{-4}, &0<x<2  \\
         0 & \text{otherwise} 
    \end{array}
\end{equation}
We know, for a continuous random variable,
\begin{align}
    E(X) &= \int_{- \infty}^{\infty} x \pr{X=x} \,dx \\
    \begin{split}
     = \int_{- \infty}^0 x \pr{X=x} \,dx + \int_0^2 x \pr{X=x} \,dx \\
     + \int_2^{\infty} x \pr{X=x} \,dx
    \end{split} \\
    &= 0 + \int_0^2 \brak{x \times 4e^{-4}} \,dx + 0 \\
    &= 4e^{-4}\brak{\frac{x^2}{2}}_0^2 \\
    &= 4e^{-4} \times 2 \\
\end{align}
\begin{equation}
    \boxed{E(X) = 0.146}
\end{equation}
\end{document}