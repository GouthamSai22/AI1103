\documentclass[journal,12pt,twocolumn]{IEEEtran}

\usepackage{setspace}
\usepackage{gensymb}
\singlespacing
\usepackage[cmex10]{amsmath}

\usepackage{amsthm}

\usepackage{mathrsfs}
\usepackage{txfonts}
\usepackage{stfloats}
\usepackage{bm}
\usepackage{cite}
\usepackage{cases}
\usepackage{subfig}
\usepackage{paralist}
\usepackage{longtable}
\usepackage{multirow}

\usepackage{enumitem}
\usepackage{mathtools}
\usepackage{steinmetz}
\usepackage{tikz}
\usepackage{circuitikz}
\usepackage{verbatim}
\usepackage{tfrupee}
\usepackage[breaklinks=true]{hyperref}
\usepackage{graphicx}
\usepackage{tkz-euclide}

\usetikzlibrary{calc,math}
\usepackage{listings}
    \usepackage{color}                                            %%
    \usepackage{array}                                            %%
    \usepackage{longtable}                                        %%
    \usepackage{calc}                                             %%
    \usepackage{multirow}                                         %%
    \usepackage{hhline}                                           %%
    \usepackage{ifthen}                                           %%
    \usepackage{lscape}     
\usepackage{multicol}
\usepackage{chngcntr}

\DeclareMathOperator*{\Res}{Res}

\renewcommand\thesection{\arabic{section}}
\renewcommand\thesubsection{\thesection.\arabic{subsection}}
\renewcommand\thesubsubsection{\thesubsection.\arabic{subsubsection}}

\renewcommand\thesectiondis{\arabic{section}}
\renewcommand\thesubsectiondis{\thesectiondis.\arabic{subsection}}
\renewcommand\thesubsubsectiondis{\thesubsectiondis.\arabic{subsubsection}}


\hyphenation{op-tical net-works semi-conduc-tor}
\def\inputGnumericTable{}                                 %%

\lstset{
%language=C,
frame=single, 
breaklines=true,
columns=fullflexible
}
\begin{document}

\newcommand{\BEQA}{\begin{eqnarray}}
\newcommand{\EEQA}{\end{eqnarray}}
\newcommand{\define}{\stackrel{\triangle}{=}}
\bibliographystyle{IEEEtran}
\raggedbottom
\setlength{\parindent}{0pt}
\providecommand{\mbf}{\mathbf}
\providecommand{\pr}[1]{\ensuremath{\Pr\left(#1\right)}}
\providecommand{\qfunc}[1]{\ensuremath{Q\left(#1\right)}}
\providecommand{\sbrak}[1]{\ensuremath{{}\left[#1\right]}}
\providecommand{\lsbrak}[1]{\ensuremath{{}\left[#1\right.}}
\providecommand{\rsbrak}[1]{\ensuremath{{}\left.#1\right]}}
\providecommand{\brak}[1]{\ensuremath{\left(#1\right)}}
\providecommand{\lbrak}[1]{\ensuremath{\left(#1\right.}}
\providecommand{\rbrak}[1]{\ensuremath{\left.#1\right)}}
\providecommand{\cbrak}[1]{\ensuremath{\left\{#1\right\}}}
\providecommand{\lcbrak}[1]{\ensuremath{\left\{#1\right.}}
\providecommand{\rcbrak}[1]{\ensuremath{\left.#1\right\}}}
\theoremstyle{remark}
\newtheorem{rem}{Remark}
\newtheorem{lemma}{Question}
\newcommand{\sgn}{\mathop{\mathrm{sgn}}}
\providecommand{\abs}[1]{\vert#1\vert}
\providecommand{\res}[1]{\Res\displaylimits_{#1}} 
\providecommand{\norm}[1]{\lVert#1\rVert}
%\providecommand{\norm}[1]{\lVert#1\rVert}
\providecommand{\mtx}[1]{\mathbf{#1}}
\providecommand{\mean}[1]{E[ #1 ]}
\providecommand{\fourier}{\overset{\mathcal{F}}{ \rightleftharpoons}}
%\providecommand{\hilbert}{\overset{\mathcal{H}}{ \rightleftharpoons}}
\providecommand{\system}{\overset{\mathcal{H}}{ \longleftrightarrow}}
	%\newcommand{\solution}[2]{\textbf{Solution:}{#1}}
\newcommand{\solution}{\noindent \textbf{Solution: }}
\newcommand{\cosec}{\,\text{cosec}\,}
\providecommand{\dec}[2]{\ensuremath{\overset{#1}{\underset{#2}{\gtrless}}}}
\newcommand{\myvec}[1]{\ensuremath{\begin{pmatrix}#1\end{pmatrix}}}
\newcommand{\mydet}[1]{\ensuremath{\begin{vmatrix}#1\end{vmatrix}}}
\numberwithin{equation}{subsection}
\makeatletter
\@addtoreset{figure}{problem}
\makeatother
\let\StandardTheFigure\thefigure
\let\vec\mathbf
\renewcommand{\thefigure}{\theproblem}
\def\putbox#1#2#3{\makebox[0in][l]{\makebox[#1][l]{}\raisebox{\baselineskip}[0in][0in]{\raisebox{#2}[0in][0in]{#3}}}}
     \def\rightbox#1{\makebox[0in][r]{#1}}
     \def\centbox#1{\makebox[0in]{#1}}
     \def\topbox#1{\raisebox{-\baselineskip}[0in][0in]{#1}}
     \def\midbox#1{\raisebox{-0.5\baselineskip}[0in][0in]{#1}}
\vspace{3cm}
\title{Assignment 2}
\author{S Goutham Sai - CS20BTECH11042}
\maketitle
\newpage
\bigskip
\renewcommand{\thefigure}{\theenumi}
\renewcommand{\thetable}{\theenumi}
Download all python codes from 
\begin{lstlisting}
https://github.com/GouthamSai22/AI1103/blob/main/Assignment2/Codes
\end{lstlisting}
%
and latex-tikz codes from 
%
\begin{lstlisting}
https://github.com/GouthamSai22/AI1103/blob/main/Assignment2/main.tex
\end{lstlisting}
\section{\textbf{Problem 70 from GATE EC}}
Let X and Y be continuous random variables with the joint probabilitiy distribution function\newline
f(x,y) = $\bigg\{ \begin{array}{lr}
    ae^{-2y}, & 0<x<y<\infty  \\
     0, &\text{otherwise} 
\end{array}$\newline
The value of $E(X|Y=2)$ is \newline
\begin{enumerate}
\item 4 
\item 3 
\item 2 
\item 1 
\end{enumerate}
\section{\textbf{Solution}}
\begin{lemma}
Given two continuous random variables X and Y, whose joint probability distribution function is\newline
\begin{equation}
    f(x,y) = \bigg\{ \begin{array}{lr}
    ae^{-2y}, & 0<x<y<\infty  \\
     0, &\text{otherwise} 
\end{array}
\end{equation}
We are asked to find the value of $E(X|Y=2)$.
\end{lemma}
\textbf{Solution}: Firstly we find the marginal distribution function for $X=x$ given $Y=y$,
\begin{equation}
    f_{x,y}(x|y) = \frac{f_{x,y}(x,y)}{f_y(y)}
\end{equation}
\begin{align}
    f_y(y) &= \int_{-\infty}^{\infty} f_{x,y}(x,y) \,dx \\
    \begin{split}
         &= \int_{-\infty}^{0} f_{x,y}(x,y) \,dx + \int_{0}^{y} f_{x,y}(x,y) \,dx \\
        &+ \int_{y}^{\infty} f_{x,y}(x,y) \,dx\\
    \end{split}\\
    &= 0 + \int_0^y ae^{-2y} \,dx + 0 \\
    f_y(y) &= ae^{-2y}y
\end{align}
Therefore,
\begin{align}
    \pr{X=x|Y=2} &= \frac{f_{x,y}(x,y)}{f_y(2)}
    \end{align}
Substituting y=2 in \equationautorefname{2.0.6}
\begin{align}
     &= \frac{ae^{-4}}{ae^{-4}2} \\
    \pr{X=x|Y=2} &= \bigg\{ \begin{array}{lr}
    \frac{1}{2}, & 0<x<y<\infty  \\
     0, &\text{otherwise} 
\end{array}
\end{align}
   

Hence, expected value of X is
\begin{align}
    E(X|Y=2) &= \int_{-\infty}^{\infty} x\pr{X=x|Y=2} \,dx \\
    \begin{split}
        &= \int_{-\infty}^{0} x\pr{X=x|Y=2} \,dx \\
        &+ \int_{0}^{2} x\pr{X=x|Y=2} \,dx\\
        &+ \int_{2}^{\infty} x\pr{X=x|Y=2} \,dx
    \end{split} \\
    &= 0+ \int_{0}^{2} x\frac{e^{4-4}}{2}\,dx  +0 \\
    &= \int_0^2 x\frac{1}{2} \,dx \\
    &= \frac{1}{2}\bigg[\frac{x^2}{2}\bigg]_0^2 \\
    &= \frac{1}{2} \frac{4}{2} 
\end{align}
\begin{equation}
    \boxed{E(X|Y=2) = 1}
\end{equation}
\end{document}