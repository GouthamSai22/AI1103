\documentclass[journal,12pt,twocolumn]{IEEEtran}

\usepackage{setspace}
\usepackage{gensymb}
\singlespacing
\usepackage[cmex10]{amsmath}

\usepackage{amsthm}

\usepackage{mathrsfs}
\usepackage{txfonts}
\usepackage{stfloats}
\usepackage{bm}
\usepackage{cite}
\usepackage{cases}
\usepackage{subfig}
\usepackage{paralist}
\usepackage{longtable}
\usepackage{multirow}

\usepackage{enumitem}
\usepackage{mathtools}
\usepackage{steinmetz}
\usepackage{tikz}
\usepackage{circuitikz}
\usepackage{verbatim}
\usepackage{tfrupee}
\usepackage[breaklinks=true]{hyperref}
\usepackage{graphicx}
\usepackage{tkz-euclide}

\usetikzlibrary{calc,math}
\usepackage{listings}
    \usepackage{color}                                            %%
    \usepackage{array}                                            %%
    \usepackage{longtable}                                        %%
    \usepackage{calc}                                             %%
    \usepackage{multirow}                                         %%
    \usepackage{hhline}                                           %%
    \usepackage{ifthen}                                           %%
    \usepackage{lscape}     
\usepackage{multicol}
\usepackage{chngcntr}

\DeclareMathOperator*{\Res}{Res}

\renewcommand\thesection{\arabic{section}}
\renewcommand\thesubsection{\thesection.\arabic{subsection}}
\renewcommand\thesubsubsection{\thesubsection.\arabic{subsubsection}}

\renewcommand\thesectiondis{\arabic{section}}
\renewcommand\thesubsectiondis{\thesectiondis.\arabic{subsection}}
\renewcommand\thesubsubsectiondis{\thesubsectiondis.\arabic{subsubsection}}


\hyphenation{op-tical net-works semi-conduc-tor}
\def\inputGnumericTable{}                                 %%

\lstset{
%language=C,
frame=single, 
breaklines=true,
columns=fullflexible
}
\begin{document}

\newcommand{\BEQA}{\begin{eqnarray}}
\newcommand{\EEQA}{\end{eqnarray}}
\newcommand{\define}{\stackrel{\triangle}{=}}
\bibliographystyle{IEEEtran}
\raggedbottom
\setlength{\parindent}{0pt}
\providecommand{\mbf}{\mathbf}
\providecommand{\pr}[1]{\ensuremath{\Pr\left(#1\right)}}
\providecommand{\qfunc}[1]{\ensuremath{Q\left(#1\right)}}
\providecommand{\sbrak}[1]{\ensuremath{{}\left[#1\right]}}
\providecommand{\lsbrak}[1]{\ensuremath{{}\left[#1\right.}}
\providecommand{\rsbrak}[1]{\ensuremath{{}\left.#1\right]}}
\providecommand{\brak}[1]{\ensuremath{\left(#1\right)}}
\providecommand{\lbrak}[1]{\ensuremath{\left(#1\right.}}
\providecommand{\rbrak}[1]{\ensuremath{\left.#1\right)}}
\providecommand{\cbrak}[1]{\ensuremath{\left\{#1\right\}}}
\providecommand{\lcbrak}[1]{\ensuremath{\left\{#1\right.}}
\providecommand{\rcbrak}[1]{\ensuremath{\left.#1\right\}}}
\theoremstyle{remark}
\newtheorem{rem}{Remark}
\newcommand{\sgn}{\mathop{\mathrm{sgn}}}
\providecommand{\abs}[1]{\vert#1\vert}
\providecommand{\res}[1]{\Res\displaylimits_{#1}} 
\providecommand{\norm}[1]{\lVert#1\rVert}
%\providecommand{\norm}[1]{\lVert#1\rVert}
\providecommand{\mtx}[1]{\mathbf{#1}}
\providecommand{\mean}[1]{E[ #1 ]}
\providecommand{\fourier}{\overset{\mathcal{F}}{ \rightleftharpoons}}
%\providecommand{\hilbert}{\overset{\mathcal{H}}{ \rightleftharpoons}}
\providecommand{\system}{\overset{\mathcal{H}}{ \longleftrightarrow}}
	%\newcommand{\solution}[2]{\textbf{Solution:}{#1}}
\newcommand{\solution}{\noindent \textbf{Solution: }}
\newcommand{\cosec}{\,\text{cosec}\,}
\providecommand{\dec}[2]{\ensuremath{\overset{#1}{\underset{#2}{\gtrless}}}}
\newcommand{\myvec}[1]{\ensuremath{\begin{pmatrix}#1\end{pmatrix}}}
\newcommand{\mydet}[1]{\ensuremath{\begin{vmatrix}#1\end{vmatrix}}}
\numberwithin{equation}{subsection}
\makeatletter
\@addtoreset{figure}{problem}
\makeatother
\let\StandardTheFigure\thefigure
\let\vec\mathbf
\renewcommand{\thefigure}{\theproblem}
\def\putbox#1#2#3{\makebox[0in][l]{\makebox[#1][l]{}\raisebox{\baselineskip}[0in][0in]{\raisebox{#2}[0in][0in]{#3}}}}
     \def\rightbox#1{\makebox[0in][r]{#1}}
     \def\centbox#1{\makebox[0in]{#1}}
     \def\topbox#1{\raisebox{-\baselineskip}[0in][0in]{#1}}
     \def\midbox#1{\raisebox{-0.5\baselineskip}[0in][0in]{#1}}
\vspace{3cm}
\title{Assignment 4}
\author{S Goutham Sai - CS20BTECH11042}
\maketitle
\newpage
\bigskip
\renewcommand{\thefigure}{\theenumi}
\renewcommand{\thetable}{\theenumi}
Download all python codes from 
\begin{lstlisting}
https://github.com/GouthamSai22/AI1103/blob/main/Assignment4/Codes
\end{lstlisting}
%
and latex-tikz codes from 
%
\begin{lstlisting}
https://github.com/GouthamSai22/AI1103/blob/main/Assignment4/main.tex
\end{lstlisting}
\section{\textbf{Problem 64 from GATE(ME) 2012}}
An automobile plant contracted to buy shock absorbers from two suppliers X and Y. X supplies 60\% and y supplies 40\% of the shock absorbers. All shock absorbers are subjected to a quality test.The ones that pass the quality test are considered reliable. Of X's shock absorbers are 96\% are reliable. Of Y's shock absorbers 72\% are reliable The probability that a randomly chosen shock absorber which is found to be reliable is made by Y is
\begin{enumerate}
    \item 0.288
    \item 0.334
    \item 0.667
    \item 0.720
\end{enumerate}
\section{\textbf{Solution}}
Let A and B be two random variables that take values from the set \{0,1\}.\newline
A:
\begin{itemize}
    \item A=0 $\rightarrow$ \text{shock absorber is from X}
    \item A=1 $\rightarrow$ \text{shock absorber is from Y}
\end{itemize}
B:
\begin{itemize}
    \item B=0 $\rightarrow$ \text{shock absorber is not reliable}
    \item B=1 $\rightarrow$ \text{shock absorber is reliable}
\end{itemize}
\begin{table}[h!]
    \centering
    \begin{tabular}{|c|c|c|}
        \hline
        $x_i$ & Description & P(A=$x_i$)\\
        \hline
        0 & Shock absorber is from X & 0.6\\
        1 & Shock absorber is from Y & 0.4\\
        \hline
    \end{tabular}
    \caption{Values taken by X}
    \label{tab:1}
\end{table}
Given,
\begin{align}
    \pr{B=1|A=0} &= 0.96\\
    \pr{B=1|A=1} &= 0.72
\end{align}
Using the fact that \pr{E|F} = $\frac{\pr{E+F}}{\pr{F}}$,
\begin{multline}
    \pr{\brak{B=1}+\brak{A=0}} = \pr{B=1|A=0} \times \\
    \pr{A=0}
\end{multline}
\begin{align}
\pr{\brak{B=1}+\brak{A=0}} &= 0.576 \\
\text{Similarly,}\label{2.0.5} \pr{\brak{B=1}+\brak{A=1}} &= 0.288 
\end{align}
Since the events (A=0) and (A=1) are mutually independent and mutually exhaustive, we can say that 
\begin{multline}
    \pr{B=1} = \pr{\brak{B=1}+\brak{A=0}} + \\
    \pr{\brak{B=1}+\brak{A=1}}.
\end{multline}
\begin{equation}\label{(2.0.7)}
    \implies \pr{B=1} = 0.864
\end{equation}
We need to find \pr{A=1|B=1}
\begin{align}
    \pr{A=1|B=1} = \frac{\pr{\brak{A=1}+\brak{B=1}}}{\pr{B=1}}
\end{align}
Substituting values from \eqnref{2.0.5} \eqnref{2.0.7)}, we get
\begin{align}
    \pr{A=1|B=1} &= \frac{0.288}{0.864} \\
    \implies \pr{A=1|B=1} &= 0.3333333 \\
    \implies \pr{A=1|B=1} &= 0.334
\end{align}
\end{document}
