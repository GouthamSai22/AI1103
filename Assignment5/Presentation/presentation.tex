\documentclass{beamer}
\usepackage{listings}
\lstset{
%language=C,
frame=single, 
breaklines=true,
columns=fullflexible
}
\usepackage{subcaption}
\usepackage{url}
\usepackage{tikz}
\usepackage{tkz-euclide} % loads  TikZ and tkz-base
%\usetkzobj{all}
\usetikzlibrary{calc,math}
\usepackage{float}
\newcommand\norm[1]{\left\lVert#1\right\rVert}
\providecommand{\brak}[1]{\ensuremath{\left(#1\right)}}
\providecommand{\abs}[1]{\vert#1\vert}
\providecommand{\fourier}{\overset{\mathcal{F}}{ \rightleftharpoons}}
\providecommand{\pr}[1]{\ensuremath{\Pr\left(#1\right)}}
\providecommand{\sbrak}[1]{\ensuremath{{}\left[#1\right]}}
\renewcommand{\vec}[1]{\mathbf{#1}}
\usepackage[export]{adjustbox}
\usepackage[utf8]{inputenc}
\usepackage{amsmath}
\usepackage{mhchem}
\usetheme{Boadilla}

\title{CSIR UGC NET Dec 2012 Q.103}
\author{S.Goutham Sai}
\institute{IITH(CSE)}
\date{\today}
\begin{document}


\begin{frame}
\titlepage
\end{frame}
\begin{frame}{Question}
\begin{block}{CSIR UGC NET Dec 2012 Q.103}
Let \{$X_n: n>0$\} and X be random variables defined on a common probability space. Further assume that \{$X_n \geq 0 \ \forall n > 0$\} and
\begin{align}
\pr{X = x} =\begin{cases}
    p, & x = 0  \\
    1-p, & x =1 \\
     0, &\text{otherwise} 
\end{cases}\end{align}
where, $0 \leq p \leq 1$. Which of the following statements are necessarily true?
\begin{enumerate}
    \item If $p = 0$ and $X_n \ce{->[d]} X$, $X_n \ce{->[p]} X$ 
    \item If $p = 1$ and $X_n \ce{->[d]} X$, then $X_n  \ce{->[p]} X$
    \item If $0<p<1$ and $X_n \ce{->[d]} X$, then $X_n \ce{->[p]} X$
    \item If $X_n \ce{->[p]} X$, then $X_n  \ce{->T[a.s]} X$
\end{enumerate}
\end{block}
\end{frame}

\begin{frame}{Definitions}
\begin{block}{Convergence in Distribution}
    For any given sequence of real-valued random variables $X_1,X_2,X_3, \dots ,X_n$ and a random variable X, 
    \begin{equation}
    \lim_{n \rightarrow \infty} F_{X_n}(x) = F_X(x)
    \end{equation}
    where $F_{X_n}$ and $F_X$ are the cumulative probability distribution functions of $X_n$ and X respectively.
\end{block}    
\end{frame}

\begin{frame}{Definitions}
\begin{block}{Convergence in Probability}
    For any given sequence of real-valued random variables $X_1,X_2,X_3, \dots ,X_n$ and a random variable X,
    \begin{align}
    \lim_{n \rightarrow \infty} \pr{|X_n - X| > \epsilon} = 0 \forall \epsilon > 0
    \end{align}
\end{block}
\begin{block}{Almost sure Convergence}
    For any given sequence of real-valued random variables $X_1,X_2,X_3, \dots ,X_n$ and a random variable X,
    \begin{align}
    \pr{\lim_{n \rightarrow \infty} X_n = X} = 1  
    \end{align}
\end{block}
\end{frame}

\begin{frame}{Portmanteaus Lemma}
\begin{block}{Portmanteaus Lemma}
    The sequence $X_1,X_2,X_3, \dots ,X_n$ converges in distribution to X if and only if
    \begin{equation}
    \limsup \pr{X_{n}\in F} \leq \pr{X \in F}
    \end{equation} for every closed set F;
\end{block}
\end{frame}

\begin{frame}{Solution}
\begin{block}{Lemma}
     Convergence in distribution implies convergence in probability if X is a constant.
\end{block}
\begin{block}{Proof}
    \begin{itemize}
        \item Let $\epsilon > 0$. Let X = c and the sequence $X_1,X_2,X_3, \dots ,X_n$ converges to X in distribution.
        \item Let \begin{align}
            &S = \{X : |X - c| > \epsilon\} \\
            \implies &\pr{|X_n - c|>\epsilon} = \pr{X_ n\in S}
        \end{align}
        \end{itemize}
\end{block}
\end{frame}

\begin{frame}{Solution}
\begin{block}{Proof}
    \begin{itemize}
        \item From Portmanteau's Lemma, 
        \begin{align}
           &\because \lim_{n \rightarrow \infty}X_n = c \\
            \implies &\limsup_{n \rightarrow \infty} \pr{X_n \in S} \leq  \pr{c \in S} \\
            &\because \pr{c \in S} = 0 \text{ (By defn)} \\
            \implies &\limsup_{n \rightarrow \infty} \pr{X_n \in S} \leq 0 \\
            \implies &\lim_{n \rightarrow \infty} \pr{X_n \in S} \leq 0\\
            \implies &\lim_{n \rightarrow \infty} \pr{X_n \in S} = 0 \text{ (Probability $\geq 0$)}
        \end{align}
        \item Thus, by definition,
        \begin{multline}
             \pr{|X_n-c|> \epsilon} = 0 \text{ for any }\epsilon > 0 \text{ given,} \\    
            \lim_{n \rightarrow \infty} F_{X_n}(x) = F_X(x)\text{ and X is constant}
        \end{multline}
    \end{itemize}
\end{block}
\end{frame}

\begin{frame}{Solution}
\begin{block}{Checking Options}
    \begin{enumerate}
    \item Given,
    \begin{align}\nonumber
        p = 0 \implies X = 1
    \end{align}
    Since X is a constant, from Lemma, we can say that option 1 is true.
    \item Given,
      \begin{align}\nonumber
        p = 1 \implies X = 0
    \end{align}
    Since X is a constant, from Lemma, we can say that option 2 is true.
    \item Given,
      \begin{align}\nonumber
        0 < p < 1 \implies X \neq 0,1 
    \end{align}
    Since X is not a constant, we can say that option 3 is false.
    
    \end{enumerate}
\end{block}
\end{frame}

\begin{frame}{Solution}
\begin{block}{Checking Options}
    \begin{enumerate}
        \item Since Convergence in probability is weaker than Almost sure convergence, we can say that option 4 is false as a weaker statement does not imply a stronger statement. 
    \end{enumerate}
    Therefore, the true statements from the options are options 1 and 2.
\end{block}    
\end{frame}

\end{document}