\documentclass[journal,12pt,twocolumn]{IEEEtran}

\usepackage{setspace}
\usepackage{gensymb}
\singlespacing
\usepackage[cmex10]{amsmath}

\usepackage{amsthm}

\usepackage{mathrsfs}
\usepackage{txfonts}
\usepackage{stfloats}
\usepackage{bm}
\usepackage{cite}
\usepackage{cases}
\usepackage{subfig}
\usepackage{paralist}
\usepackage{longtable}
\usepackage{multirow}
\usepackage[latin1]{inputenc}

\usepackage{amsmath}

\usepackage{amsfonts}

\usepackage{mhchem}

\RequirePackage{amsmath,amssymb,latexsym}

\usepackage{enumitem}
\usepackage{mathtools}
\usepackage{steinmetz}
\usepackage{tikz}
\usepackage{circuitikz}
\usepackage{verbatim}
\usepackage{tfrupee}
\usepackage[breaklinks=true]{hyperref}
\usepackage{graphicx}
\usepackage{tkz-euclide}

\usetikzlibrary{calc,math}
\usepackage{listings}
    \usepackage{color}                                            %%
    \usepackage{array}                                            %%
    \usepackage{longtable}                                        %%
    \usepackage{calc}                                             %%
    \usepackage{multirow}                                         %%
    \usepackage{hhline}                                           %%
    \usepackage{ifthen}                                           %%
    \usepackage{lscape}     
\usepackage{multicol}
\usepackage{chngcntr}

\DeclareMathOperator*{\Res}{Res}

\renewcommand\thesection{\arabic{section}}
\renewcommand\thesubsection{\thesection.\arabic{subsection}}
\renewcommand\thesubsubsection{\thesubsection.\arabic{subsubsection}}

\renewcommand\thesectiondis{\arabic{section}}
\renewcommand\thesubsectiondis{\thesectiondis.\arabic{subsection}}
\renewcommand\thesubsubsectiondis{\thesubsectiondis.\arabic{subsubsection}}


\hyphenation{op-tical net-works semi-conduc-tor}
\def\inputGnumericTable{}                                 %%

\lstset{
%language=C,
frame=single, 
breaklines=true,
columns=fullflexible
}
\newtheorem{theorem}{Theorem}
\newtheorem{lemma}[theorem]{Lemma}
\begin{document}

\newcommand{\BEQA}{\begin{eqnarray}}
\newcommand{\EEQA}{\end{eqnarray}}
\newcommand{\define}{\stackrel{\triangle}{=}}
\bibliographystyle{IEEEtran}
\raggedbottom
\setlength{\parindent}{0pt}
\providecommand{\mbf}{\mathbf}
\providecommand{\pr}[1]{\ensuremath{\Pr\left(#1\right)}}
\providecommand{\qfunc}[1]{\ensuremath{Q\left(#1\right)}}
\providecommand{\sbrak}[1]{\ensuremath{{}\left[#1\right]}}
\providecommand{\lsbrak}[1]{\ensuremath{{}\left[#1\right.}}
\providecommand{\rsbrak}[1]{\ensuremath{{}\left.#1\right]}}
\providecommand{\brak}[1]{\ensuremath{\left(#1\right)}}
\providecommand{\lbrak}[1]{\ensuremath{\left(#1\right.}}
\providecommand{\rbrak}[1]{\ensuremath{\left.#1\right)}}
\providecommand{\cbrak}[1]{\ensuremath{\left\{#1\right\}}}
\providecommand{\lcbrak}[1]{\ensuremath{\left\{#1\right.}}
\providecommand{\rcbrak}[1]{\ensuremath{\left.#1\right\}}}
\theoremstyle{remark}
\newtheorem{rem}{Remark}
\newcommand{\sgn}{\mathop{\mathrm{sgn}}}
\providecommand{\abs}[1]{\vert#1\vert}
\providecommand{\res}[1]{\Res\displaylimits_{#1}} 
\providecommand{\norm}[1]{\lVert#1\rVert}
%\providecommand{\norm}[1]{\lVert#1\rVert}
\providecommand{\mtx}[1]{\mathbf{#1}}
\providecommand{\mean}[1]{E[ #1 ]}
\providecommand{\fourier}{\overset{\mathcal{F}}{ \rightleftharpoons}}
%\providecommand{\hilbert}{\overset{\mathcal{H}}{ \rightleftharpoons}}
\providecommand{\system}{\overset{\mathcal{H}}{ \longleftrightarrow}}
	%\newcommand{\solution}[2]{\textbf{Solution:}{#1}}
\newcommand{\solution}{\noindent \textbf{Solution: }}
\newcommand{\cosec}{\,\text{cosec}\,}
\providecommand{\dec}[2]{\ensuremath{\overset{#1}{\underset{#2}{\gtrless}}}}
\newcommand{\myvec}[1]{\ensuremath{\begin{pmatrix}#1\end{pmatrix}}}
\newcommand{\mydet}[1]{\ensuremath{\begin{vmatrix}#1\end{vmatrix}}}
\numberwithin{equation}{subsection}
\makeatletter
\@addtoreset{figure}{problem}
\makeatother
\let\StandardTheFigure\thefigure
\let\vec\mathbf
\renewcommand{\thefigure}{\theproblem}
\def\putbox#1#2#3{\makebox[0in][l]{\makebox[#1][l]{}\raisebox{\baselineskip}[0in][0in]{\raisebox{#2}[0in][0in]{#3}}}}
     \def\rightbox#1{\makebox[0in][r]{#1}}
     \def\centbox#1{\makebox[0in]{#1}}
     \def\topbox#1{\raisebox{-\baselineskip}[0in][0in]{#1}}
     \def\midbox#1{\raisebox{-0.5\baselineskip}[0in][0in]{#1}}
\vspace{3cm}
\title{Assignment 5}
\author{S Goutham Sai - CS20BTECH11042}
\maketitle
\newpage
\bigskip
\renewcommand{\thefigure}{\theenumi}
\renewcommand{\thetable}{\theenumi}
Download all python codes from 
\begin{lstlisting}
https://github.com/GouthamSai22/AI1103/blob/main/Assignment5/Codes
\end{lstlisting}
%
and latex-tikz codes from 
%
\begin{lstlisting}
https://github.com/GouthamSai22/AI1103/blob/main/Assignment5/main.tex
\end{lstlisting}
\section{\textbf{Problem 103 from CSIR UGC NET - Dec 2012}}
Let \{$X_n: n>0$\} and X be random variables defined on a common probability space. Further assume that \{$X_n \geq 0 \forall n > 0$\} and \newline
\pr{X = x} = $\bigg\{ \begin{array}{lr}
    p, & x = 0  \\
    1-p, & x =1 \\
     0, &\text{otherwise} 
\end{array}$\newline
where, $0 \leq p \leq 1$. Which of the following statements are necessarily true?
\begin{enumerate}
    \item If $p = 0$ and $X_n \ce{->[d]} X$, $X_n \ce{->[p]} X$ 
    \item If $p = 1$ and $X_n \ce{->[d]} X$, then $X_n  \ce{->[p]} X$
    \item If $0<p<1$ and $X_n \ce{->[d]} X$, then $X_n \ce{->[p]} X$
    \item If $X_n \ce{->[p]} X$, then $X_n  \ce{->T[a.s]} X$
\end{enumerate}
\section{\textbf{Solution}}
For any given sequence of real-valued random variables $X_1,X_2,X_3, \dots ,X_n$ and a random variable X, let us look at some definitions
\begin{enumerate}
    \item \textbf{Convergence in Distribution or Weak convergence}:
\begin{equation}
    \lim_{n \rightarrow \infty} F_n(X_n) = F(X)
\end{equation}
where F is the cumulative probability distribution function.
\item \textbf{Convergence in Probability}:
\begin{align}
    \lim_{n \rightarrow \infty} \pr{|X_n - X| > \epsilon} = 0 \text{, for all } \epsilon > 0
\end{align}
This is stronger than the convergence in distribution but weaker than Almost sure convergence.
\item \textbf{Almost sure Convergence}:
\begin{align}
  \pr{\lim_{n \rightarrow \infty} X_n = X} = 1  
\end{align}
This type of convergence is stronger than both convergence in distribution and probability.
\end{enumerate}
\begin{itemize}
    \item In general, stronger statements imply weaker statements but not vice versa, i.e. Convergence in probability implies convergence in distribution and Almost sure convergence implies convergence in probability.\newline
\end{itemize}
\begin{lemma} Convergence in distribution implies convergence in probability if X is a constant.\end{lemma}
    \textbf{Proof}:
    \begin{itemize}
        \item Let $\epsilon > 0$. Let X = c and the sequence $X_1,X_2,X_3, \dots ,X_n$ converges to X in distribution.
        \item Let \begin{align}
            &B_c( \epsilon ) \rightarrow \text{open ball about c of radius } \epsilon \\
            &B_c( \epsilon )' \rightarrow \text{complement of } B_c( \epsilon ) \\
            \implies &\pr{|X_n - c|>\epsilon} = \pr{X_ n\in B_c( \epsilon )'}
        \end{align}
        \item From the Portmanteaus lemma,\begin{lemma}
             The sequence $X_1,X_2,X_3, \dots ,X_n$ converges in distribution to X if and only if \begin{equation}
                  \limsup \pr{X_{n}\in F} \leq \pr{X \in F}
             \end{equation} for every closed set F;
        \end{lemma} 
        \begin{align}
           \text{Since }\lim_{n \rightarrow \infty}X_n &= c \\
            \implies \limsup_{n \rightarrow \infty} \pr{X_n \in B_c( \epsilon )'} &\leq  \pr{c \in B_c( \epsilon )'} \\
            \pr{c \in B_c( \epsilon )'} &= 0 \text{ (By defn)}
        \end{align}
        \item Thus,
        \begin{align}
            \lim_{n \rightarrow \infty}\pr{|X_n - c|>\epsilon} &= \lim_{n \rightarrow \infty}\pr{X_n \in B_c( \epsilon )'} \\
            & \leq \limsup_{n \rightarrow \infty}\pr{X_n \in B_c( \epsilon )'} \\
            &\leq \pr{c \in B_c( \epsilon )'} \\
            &\leq 0 \\
            &= 0 
        \end{align}
        (Since probability cannot be negative)
        \item Thus, by definition,
        \begin{multline}
             \pr{|X_n-c|> \epsilon} = 0 \text{ for any }\epsilon > 0 \text{ given,} \\    
            \lim_{n \rightarrow \infty} F_n(X_n) = F(X)\text{ and X is constant}
        \end{multline}
    \end{itemize}

Let us look at each option one after another.
\begin{enumerate}
    \item Given,
    \begin{align}\nonumber
        p = 0 \implies X = 1
    \end{align}
    Since X is a constant, from Lemma \lemmaref{1}, we can say that option 1 is true.
    \item Given,
      \begin{align}\nonumber
        p = 1 \implies X = 0
    \end{align}
    Since X is a constant, from Lemma \lemmaref{1}, we can say that option 2 is true.
    \item Given,
      \begin{align}\nonumber
        0 < p < 1 \implies X \neq 0,1
    \end{align}
    Since X is not a constant, we can say that option 3 is false.
    \item Since Convergence in probability is weaker than Almost sure convergence, we can say that option 4 is false as a weaker statement does not imply a stronger statement.
\end{enumerate}

\end{document}
